\documentclass[a4paper]{article}
\usepackage{amsmath}

\DeclareMathOperator{\WD}{W}
\DeclareMathOperator{\SD}{D}


\title{Constructive hierarchies}
\author{Louis Warren}

\begin{document}
\maketitle

\section{Intuition}

\paragraph{}
Consider directed graphs in which the existence of only some of the edges are
known. Moreover, suppose that it is known that some vertices are disconnected.
It may be possible to recover precisely which edges exist from this information;
on the other hand, the existence of some edges may be undecidable.

\section{Formalisation}

\paragraph{}
A \emph{hierarchy} $H$ is a triple $H = (V, E, S)$, where $V$ is a set of
\emph{vertices}, $E \subseteq V \times V$ is a set of \emph{edges}, and $S
\subseteq V \times V$ is a set of \emph{separations}.

\paragraph{}
Consider vertices $x, y, z \in V$ in some hierarchy $H = (V, E, S)$. We say $x$
is \emph{connected} to $y$, written $x \geq y$, if there is a finite set of edges
\[
	\{(x, v_1), (v_1, v_2), (v_2, v_3), \dotsc (v_{n-1}, v_n), (v_n, y)\}
\]
in $E$ (and trivially $x \geq x$). Note that this is equivalent to $x$ being
connected to $y$ in the digraph $(V, E)$. We call $(V, E)$ the \emph{strong
digraph} of $H$, or $\SD(H)$.

\paragraph{}
The \emph{weak digraph} of $H$ is $\WD(H) = (V, \widetilde{E}, S)$, where
\[
	\widetilde{E} = \{ e \in V \times V \mid
		\forall_{(x, y) \in S}\left(
			x \not\geq y \text{ in } (V, E \cup\{e\}, S)
		\right)
	\}
. \]
The hierarchy is \emph{consistent} if $\SD(H)$ is a subgraph of $\WD(H)$.  If
there is a separation of $x$ and $y$, then there will be no consistent
supergraph of $\SD(H)$ in which $x$ and $y$ are connected.

\end{document}
